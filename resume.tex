%!TEX program = xelatex

\documentclass[]{deedy}

% Package Imports
% ---------------
% For using symbols: links, email
\usepackage{marvosym}

% For customizing underline
\usepackage{soul}

% Set underline policy
\setul{2.5pt}{.4pt} % 1pt below contents

% Define color
\definecolor{light-gray}{gray}{0.95}

% Definitions
% -----------
\newcommand{\linksymbol}{\Mundus}
\newcommand{\linkstyle}[1]{\textbf{\ul{#1}}}
\newcommand{\grade}[1]{{\fboxsep=0.4ex\colorbox{light-gray}{\small{{#1}}}}}

% Update href command to include custom styling
\let\oldhref\href
\renewcommand{\href}[2]{%
\oldhref{#1}{\linkstyle{#2}}%
}

% Start of document
\begin{document}

% Last updated
% \lastupdated

% Define title
\namesection{Aditya}{Basu}{\linksymbol\  \href{http://www.adityabasu.me}{http://adityabasu.me} |
%\Letter\ \href{mailto:ab.aditya.basu@gmail.com}{ab.aditya.basu@gmail.com} |
\Letter\ \href{mailto:azb254@cse.psu.edu}{azb254@cse.psu.edu} |
\Mobilefone\ \texttt{+}1 (814) 862-8300
}

%%%%%%%%%%%%%%%%%%%%%%%%%%%%%%%%%%%%%%
% Column #1
%%%%%%%%%%%%%%%%%%%%%%%%%%%%%%%%%%%%%%

\begin{minipage}[t]{0.33\textwidth} 

\section{Education}
\subsection{Penn State Univ. (PSU)}
\descript{PhD in Computer Science}
\location{On Going | University Park, PA \\
Cum. GPA: 3.83 / 4
}
\sectionsep

\subsection{Dhirubhai Ambani (DA-IICT)}
\descript{B.Tech. in Information and Communication Technology}
\location{Grad. Jan 2015 | Gujarat, India \\
Cum. GPA: 8.55 / 10
%Major GPA: 9.52 / 10
}
\sectionsep

\subsection{Research Interests}
Systems Security \textbullet{} Operating Systems

\section{Coursework}
\subsection{Graduate}
Computer Security \grade{A}\\
Hardware Security \grade{A}\\
Operating System Design \grade{A}\\
%Computer Networks \grade{A-}\\
\sectionsep

\subsection{Undergraduate}
System and Network Security \grade{AA}\\
Systems Software \grade{AA}\\
Operating Systems \grade{AA}\\
Embedded Hardware Design \grade{AA}\\
%Formal Specification \& Verification \grade{AA}
%Computer Networks \grade{BB}\\
%Approaches to Semantic Web \grade{AB}\\
%Fuzzy-Neural Systems \grade{BB}\\
\sectionsep

%\subsection{Graduate-level}
%Approaches to Semantic Web \grade{AB}\\
%Fuzzy-Neural Systems \grade{BB}\\
%\sectionsep

\subsection{Select Projects}
\subsubsection*{Side-Channel (SC) on RSA}
{\small Verilog | 10 hrs./week | 8 weeks}\\[1ex]
Implemented RSA on FPGA and did SCA to find the private key.\\[1.5ex]

\subsubsection*{Key-Value Server-Client}
{\small C++ | 10 hrs./week | 8 weeks}\\[1ex]
Coded a sync \& an async version of KV server, which was highly available.

%\subsubsection*{Personal Web Scrapbook}
%{\small Django | 10 hrs./week | 8 weeks | \href{https://bitbucket.org/mitthu/capsule/src/}{CODE}}\\[1ex]
%Scrapes and saves webpages to a central server via chrome extension.\\[1.5ex]
%
%\subsubsection*{Note-Sync Tool}
%{\small C | 10 hrs./week | 4 weeks | \href{https://github.com/mitthu/note-sync}{CODE}}\\[1ex]
%Watches a note via inotify (linux) / kqueues (mac) and syncs it on change.

\section{Skills}
\subsection{Programming}
\location{Over 5000 lines:}
C \textbullet{} Python \textbullet{} Django \textbullet{} Bash \textbullet{} \LaTeX \textbullet{} Puppet \textbullet{} Ansible \textbullet{} Java \textbullet{} HTML
\sectionsep

\location{Over 1000 lines:}
ARM Assembly \textbullet{} x86 Assembly \textbullet{} CSS \textbullet{} Javascript
\sectionsep

\subsection{Technologies}
Git \textbullet{} Linux \textbullet{} Mac \textbullet{} GNU make \textbullet{} Docker %MySQL

\section{Links} 
Github:// \href{https://github.com/mitthu}{mitthu} \textbullet{} 
LinkedIn://  \href{https://www.linkedin.com/in/mitthu}{mitthu}

\section{Hobbies}
Programming circuits \textbullet{} Karate
%Photography \textbullet{} Travelling \textbullet{} Painting \textbullet{} Programming circuits

\end{minipage}
%%%%%%%%%%%%%%%%%%%%%%%%%%%%%%%%%%%%%%
% Column #2
%%%%%%%%%%%%%%%%%%%%%%%%%%%%%%%%%%%%%%
%\hfill
\vrule{}
\hfill
\begin{minipage}[t]{0.65\textwidth}
\section{Refereed Publication}
\vspace{\topsep} % Hacky fix for awkward extra vertical space
\begin{tightemize}
	% Dates: 16\textsuperscript{th}-20\textsuperscript{th} Dec.
	\item \textbf{Aditya Basu}, Anish Mathuria, Nagendra Chowdary. \textbf{Automatic Generation of Compact Alphanumeric Shellcodes for x86}. 10\textsuperscript{th} International Conference on Information Systems Security (Hyderabad, India), LNCS 8880, Springer, pp. 399-410, 2014. (accept. rate: $19\%$ or $25/129$) [ \href{https://dl.dropboxusercontent.com/u/9020146/resources/alpha-x86.pdf}{pdf} | \href{http://dx.doi.org/10.1007/978-3-319-13841-1_22}{doi} | \href{https://dl.dropboxusercontent.com/u/9020146/resources/ppt/alpha-x86.pdf}{ppt} | \href{https://bitbucket.org/mitthu/alpha_loaders/src}{code} ]
\end{tightemize}

\section{Research Experience}
\subsection{CFI-enforcement on Linux}
\location{Aug. 2017 – current | 20 hrs./week | Supervisor: Trent Jaeger}
% \vspace{\topsep} % Hacky fix for awkward extra vertical space
\begin{tightemize}
\item Intrumenting the Linux kernel to generate CFI policy and enforce it at runtime.
\end{tightemize}
\sectionsep

\subsection{Compact Alphanumeric Shellcodes for x86}
\location{Feb. 2014 – Apr. 2014 | 25 hrs./week | Supervisor: Anish Mathuria | \href{https://dl.dropboxusercontent.com/u/9020146/resources/reports/major_project.pdf}{REPORT}}
% \vspace{\topsep} % Hacky fix for awkward extra vertical space
\begin{tightemize}
\item Designed and implemented two new encoding schemes for producing compact alphanumeric shellcodes for x86, as compared to Rix and Wever.
\end{tightemize}
\sectionsep

%\subsection{Alphanumeric Shellcode Generator for ARM}
%\location{Nov. 2013 – Jan. 2014 | 25 hrs./week | Supervisor: Anish Mathuria}
%% \vspace{\topsep} % Hacky fix for awkward extra vertical space
%\begin{tightemize}
%\item Helped in testing out an alphanumeric generator for ARM and wrote report on the internals of the tool. The tool was designed by another UG student.
%\end{tightemize}
%\sectionsep

%\runsubsection{Credit Scheduler}
%\descript{| Summer Internship}
%\location{Jun. 2013 – Aug. 2013 | 20 hrs./week | Supervisor: Sanjay Chaudhary}
%% \vspace{\topsep} % Hacky fix for awkward extra vertical space
%\begin{tightemize}
%\item Studied scheduling algorithms used in virtualisation and tested them against various workloads.
%\end{tightemize}

\section{Experience}
\runsubsection{Course: Operating Systems}
\descript{| Teaching Assistant}
\location{Jan. 2018 - Current, Aug. 2016 - Dec. 2017 | 20 hrs./week | PSU, PA, USA}
\sectionsep

\runsubsection{Media.net}
\descript{| System Operations }
\location{July 2014 - July 2016 | Mumbai, India}
% \vspace{\topsep} % Hacky fix for awkward extra vertical space
\begin{tightemize}
\item Automating and managing infrastructure serving >100 million reqs./day.
\item Took two day training session on 
\begin{enumerate*}[label=(\roman*)]
  \item \textit{Advanced Linux}, and
  \item \textit{Networking}
\end{enumerate*}
 for new recruits.
\end{tightemize}
\sectionsep

\runsubsection{Course: Systems Software}
\descript{| Teaching Assistant}
\location{Jan. 2014 - Apr. 2014 | 10 hrs./week | DA-IICT, Gujarat, India}
\sectionsep

\runsubsection{Admission Portal of DA-IICT}
\descript{| Software Developer}
\location{Jun-Jul 2013, Jun-Jul 2014 \& Jun-Jul 2015 | 30 hrs./week | Gujarat, India}
\begin{tightemize}
\item Created the admission portal of the university using Django (python).
\item The portal generates merit-lists and wait-lists of candidates based on their stream preferences and merit ranks. 
\item The portal also handles all emails communications and provides a web interface for the candidates and the admissions team.
\end{tightemize}
\sectionsep

\runsubsection{Jagrut Nagrik (NGO)}
\descript{| Rural Internship}
\location{Dec. 2011 | 40 hrs./week (full-time) | Ramosana, Gujarat, India | \href{https://dl.dropboxusercontent.com/u/9020146/resources/reports/rural_internship.pdf}{REPORT}}

\section{Awards}
\begin{tabular}{rll}
2014 & honor & Highest CGPA in UG among wards of SVB trust\\
2011 & honor & Highest percentage in HSC among wards of SVB trust\\
2010 & top 2.5\%ile & All India Engineering Entrance Examination (AIEEE)\\
2010 & 97.5 / 120 & Gujarat Common Entrance Test (GUJCET)\\
2007 & certified `A' grade & Intermediate Drawing, Gujarat State Examination Board\\
\end{tabular}

\section{Extracurriculars}
% \vspace{\topsep} % Hacky fix for awkward extra vertical space
\begin{tabular}{ll}
2013 & Founder and coordinator of  \href{http://lpdaiict.wordpress.com/}{LINUX PACK CLUB}, DA-IICT.\\
2012 & Attended \textbf{Programming System on Chip-3 (PSoC-3) Workshop}.\\
2012 & Attended \textbf{Research Workshop on Intro. to Graph and Geometric Algo.}
\end{tabular}

\end{minipage}
\end{document}