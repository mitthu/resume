% Copyright of Aditya Basu, 27-Aug-2019.
\documentclass[margin]{res}

\newif\ifdraft % skip versioning

%\drafttrue % comment out for actual use

% the margin option causes section titles to appear to the left of body text 
\textwidth=5.2in % increase textwidth to get smaller right margin

% Import packages
% ---------------
\usepackage{hyperref}
\usepackage{scrextend}  % labelling env

% Setup fonts
% -----------
\usepackage{fontspec}
\defaultfontfeatures{Mapping=tex-text}

\setmainfont[
	Color=primary,
	Path=../fonts/lora/,
	BoldItalicFont=Lora-BoldItalic,
	BoldFont=Lora-Bold,
	ItalicFont=Lora-Italic
]{Lora-Regular}

\setsansfont[
	Color=primary,
	Path=../fonts/lato/,
	BoldItalicFont=Lato-RegIta,
	BoldFont=Lato-Reg,
	ItalicFont=Lato-RegIta
]{Lato-Reg}

% Custom lists, package: enumitem
\usepackage[inline]{enumitem}
\newlist{itemizeexp}{itemize}{1} % itemize experience
\setlist[itemizeexp]{
    label       = \textbullet,
    rightmargin = 2cm,
    itemsep     = -2pt % reduce space between items
}

% setup header and footer
\usepackage{fancyhdr}
\renewcommand{\headrulewidth}{0pt}
\renewcommand{\footrulewidth}{0.4pt}

\ifdraft
	\newcommand{\version}{}
\else
	\newcommand{\version}{\input|"make -f ../Makefile --silent version"}
\fi

\pagestyle{fancy}
\fancyhead{} % clear all header fields
\fancyfoot{} % clear all footer fields
\lfoot{\version}
\rfoot{\thepage}

% commands
\newcommand{\daiict}{\textsc{DAIICT}}
\newcommand{\psu}{\textsc{PennState}}
\newcommand{\doi}[1]{\href{http://dx.doi.org/#1}{#1}}

% main document
%%%%%%%%%%%%%%%%%%%%%%%%%%%%%%%%%%%%%%%%%%%%%%%%%%%%%%%%%%%%%%%%%%%%%%%%%
\begin{document}

\raggedright
\name{Aditya Basu\\[12pt]} % the \\[12pt] adds a blank line after name

\address{
    blog:// \href{http://adityabasu.me}{adityabasu.me}\\
    github:// \href{https://github.com/mitthu}{mitthu}\\
    linkedIn://  \href{https://www.linkedin.com/in/mitthu}{mitthu}
}

\address{
    W106 Westgate Building\\
    University Park, PA 16802\\
    (814) 862-8300\\
    \href{mailto:aditya.basu@psu.edu}{aditya.basu@psu.edu}
}

\begin{resume}
% todos:
% * add lectures taken on behalf of Trent
% * add a comment on rural internship
% * move all additional stuff from resume to cv
% * update the resume to only have the necessary information
% * link my testimonial for DAIICT's placement cell

\section{About}
My research interests are in \emph{Systems Security} and \emph{Operating Systems}.
Currently I am working on generating control-flow graphs of the Linux kernel and its 
user-space programs using Intel\textsuperscript{\textregistered} Processor Trace.
The goal is to explore new metrics for system monitoring in order to defend against zero days.

\section{Education}

\textbf{PhD Student} in \emph{Computer Science} \hfill (exp.) May 2022\\
at \emph{Pennsylvania State University}, PA, USA\\
GPA: 3.85 (of 4)

\textbf{B.Tech.} in \emph{Information and Communication Technology} \hfill August 2014\\
Dhirubhai Ambani Institute (\daiict), Gujarat, India\\
GPA: 9.52 (of 10) in major,
8.55 overall 

\section{Coursework}

\begin{labeling}{\psu++}

\item[{\bfseries\psu}]
	$\longrightarrow$ Computer Security (A)\\
	$\longrightarrow$ Introduction to Hardware Security (A)\\
	$\longrightarrow$ Operating System Design (A)\\
	$\longrightarrow$ Fundamentals of Computer Architecture (A)\\
	$\longrightarrow$ Public Cloud Computing (A)\\
	$\longrightarrow$ Introduction to Distributed Computing (A)\\
	$\longrightarrow$ Compiler Construction (A-)\\[2ex]

\item[{\bfseries\daiict}]
	$\longrightarrow$ System and Network Security (A)\\
	$\longrightarrow$ Systems software (A)\\
	$\longrightarrow$ Operating systems (A)\\
	$\longrightarrow$ Embedded Hardware Design (A)\\
	$\longrightarrow$ Formal Specification \& Verification (A)

\end{labeling}

\section{Referred Publications}

\begin{labeling}{YYYY}
\item[2020]
    \textbf{Automatic Generation of Compact Printable Shellcodes for x86}\\
    Dhrumil Patel, \emph{Aditya Basu}, Anish Mathuria.\\
    In \emph{14\textsuperscript{th} USENIX Workshop on Offensive Technologies ({WOOT})}.\\
    {\itshape (acceptance rate: $33.33\%$, or $12/36$)}
    %\url{https://www.usenix.org/conference/woot20/presentation/patel} \\

\item[2020]
    \textbf{Hardware Assisted Buffer Protection Mechanisms for Embedded RISC-V}\\
    Asmit De, \emph{Aditya Basu}, Swaroop Ghosh, Trent Jaeger.\\
    In \emph{IEEE Transactions on Computer-Aided Design of Integrated Circuits and Systems}.
    {\itshape (doi: \doi{10.1109/TCAD.2020.2984407}, impact factor = 2.168)}

\item[2019]
    \textbf{FIXER: Flow Integrity Extensions in Embedded RISC-V}\\
    Asmit De, \emph{Aditya Basu}, Swaroop Ghosh and Trent Jaeger.\\
    In \emph{Proceedings of Design, Automation and Test in Europe (DATE)}.\\
    {\itshape (doi: \doi{10.23919/DATE.2019.8714980}, acceptance rate: $24\%$)}

\item[2014]
    \textbf{Automatic Generation of Compact Alphanumeric Shellcodes for x86}\\
    \emph{Aditya Basu}, Anish Mathuria, Nagendra Chowdary.\\
    In Proceedings of 10\textsuperscript{th} International Conference on Information Systems Security (ICISS).\\
    {\itshape (doi: \doi{10.1007/978-3-319-13841-1\_22}, acceptance rate: $19\%$, or $25/129$)}
    % code: \url{https://bitbucket.org/mitthu/alpha_loaders/src}
    % pdf: \url{https://dl.dropboxusercontent.com/u/9020146/resources/alpha-x86.pdf}
    % ppt: \url{https://dl.dropboxusercontent.com/u/9020146/resources/ppt/alpha-x86.pdf}
    % book: LNCS 8880, Springer, pp. 399-410
    % location: Hyderabad, India.
\end{labeling}

\section{Industry Experience}
%{\bf Title,} Company, City, State, Country \hfill Year
%\begin{itemizeexp}
%    \item Template
%\end{itemizeexp}

{\bf Software Engineering Intern,} Google, Cambridge, MA, USA
\hfill Summer  2019 % May 13 - August 16
\begin{itemizeexp}
    \item Added support for Intel VT-d to the Akaros kernel from UC Berkeley.
    This allows any PCI/PCIe device to be placed in the address space of a process or a VM.
    \item Wrote a CBDMA (DMA accelerator on Intel, aka IOAT) driver for Akaros.
    This was used to test the VT-d support.
    \item \url{https://github.com/brho/akaros/commits?author=mitthu}
    % \item Language: C
\end{itemizeexp}

{\bf Product Security Intern,} NIO, San Jose, CA, USA
\hfill Summer 2018 % May 2018 - Aug. 2018
\begin{itemizeexp}
    \item Worked on securing and pen-testing the ES8's (SUV) firmware and
    OBD-II diagnostics port.
    \item Wrote a driver for an on-board network switch and created patches
    to fix the discovered vulnerabilities.
\end{itemizeexp}

{\bf System Operations,} Media.net, Mumbai, India
\hfill 2014 - 2016 % July 2014 - July 2016
\begin{itemizeexp}
    \item Automated and managed their web crawling infrastructure serving
    $>$100 million reqs./day.
    \item Helped with recruiting and also took training sessions on 
    \begin{enumerate*}[label=(\roman*)]
      \item \textit{Advanced Linux}, and
      \item \textit{Networking}
    \end{enumerate*}
    for new recruits.
\end{itemizeexp}

{\bf Software Developer Intern,} \daiict, Gandhinagar, India
\hfill Summer 2013 \& 2014 % Jun-Jul 2013, Jun-Jul 2014, Jun-Jul 2015
\begin{itemizeexp}
    \item Created the admission portal of the university. The portal generated
    merit-lists and wait-lists of candidates based on their stream preferences
    and scores. 
    \item The portal also handled all emails communications, provided a
    web interface for the candidates and the admissions team.
    \item Framework used: Django (python)
\end{itemizeexp}

\section{Projects}
% add dates?
{\bfseries Side-Channel (SC) on RSA}\hfill
{\small Verilog | 10 hrs./week | 8 weeks}\\
Implemented RSA on an FPGA and did side-channel analysis to find the private key.\\[2ex]

{\bfseries Key-Value Server-Client}\hfill
{\small C++ | 10 hrs./week | 8 weeks}\\
A highly available key-value server supporting multiple execution models:
single-threaded asynchronous model \& multi-threaded synchronous model.\\[2ex]

{\bfseries Personal Web Scrapbook}\hfill
{\small Django | 10 hrs./week | 8 weeks | \href{https://bitbucket.org/mitthu/capsule/src/}{CODE}}\\
Scrapes and saves webpages to a central server via a chrome extension.\\[2ex]

{\bfseries Note-Sync Tool}\hfill
{\small C | 10 hrs./week | 4 weeks | \href{https://github.com/mitthu/note-sync}{CODE}}\\
Watches a note via inotify (on Linux) / kqueues (on Mac OS X) and syncs it upon modification.

\section{Assistantships} 
\begin{description}
    \item[Research Assistant,] \psu \hfill Fall'18 - current
    \item[Teaching Assistant,] \psu\ -- CMPSC473: Operating Systems \hfill Spring '18 % Jan. 2018 - May 2018
    \item[Research Assistant,] \psu \hfill 2017
    \item[Teaching Assistant,] \psu\ -- CMPSC473: Operating Systems \hfill Fall '16 % Aug. 2016 - Dec. 2016 
    \item[Teaching Assistant,] \daiict\ -- IT215: Systems Software \hfill Spring '14 % Jan. 2014 - Apr. 2014 
\end{description}

\section{Skills}
% Have used at some point:
%    ARM assembly, Javascript, MySQL
%    Matlab, Altera (FPGA), Xilinx (FPGA), Verilog
%    Postgres, MariaDB, Redis, MongoDB
%    Mesos, Hadoop, HBase
%    Wireshark
\begin{description}
    \item[>10k lines:]
        C \textbullet{}
        Python \textbullet{}
        bash \textbullet{}
        \LaTeX\ (macros) \textbullet{}
        HTML

    \item[5k --- 10k lines:]
        Django \textbullet{}
        Puppet \textbullet{}
        Ansible \textbullet{}
        Java \textbullet{}
        CSS

    \item[Utilities:]
        make \textbullet{}
        git \textbullet{}
        Docker \textbullet{}
        strace \textbullet{}
        gdb

    \item[Others:]
        Linux \textbullet{}
        Mac OS X \textbullet{}
        Markdown \textbullet{}
        IDA Hex-Rays
\end{description}

\section{Awards and Certifications}
% add GRE and TOEFL scores
\begin{labeling}{YYYY}
\item[2014] Highest CGPA in baccalaureate among wards of SVB trust
\item[2011] Highest percentage in HSC among wards of SVB trust
\item[2010] In \emph{top 2.5 percentile} of All India Engineering Entrance Examination (AIEEE)
\item[2010] \emph{97.5/120} in Gujarat Common Entrance Test (GUJCET)
\item[2007] Certified `A' grade in Intermediate Drawing, Gujarat State Exam. Board
\end{labeling}

\section{Extracurriculars}

\begin{labeling}{YYYY-YY}
\item[2020-21] President of \emph{Social Dance Club} and \emph{Shotokan Karate-do Club} at \psu.
\item[2019-20] Coordinator of \emph{Argentine Tango (of Social Dance Club)} at \psu. % Fall 2019 and Spring 2020
\item[2013] Started the \href{http://lpdaiict.wordpress.com/}{\itshape Linux Pack Club} at \daiict.
\end{labeling}

% \section{Attended}
% \item DEFCON 26, 2019
% \item Attended \emph{Programming System on Chip-3 (PSoC-3) Workshop}.\hfill 2012
% \item Attended \emph{Research Workshop on Intro. to Graph and Geometric Algo.}\hfill 2012

\section{Hobbies}
    Karate \textbullet{}
    Argentine Tango

\end{resume} 
\end{document} 
